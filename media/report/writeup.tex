\documentclass[10pt,twocolumn,letterpaper]{article}
%\usepackage[latin1]{inputenc}

%\usepackage{url}
%\usepackage{booktabs}

\usepackage{iccv}
\usepackage{times}
\usepackage{epsfig}
\usepackage{graphicx}
\usepackage{grffile}
\usepackage{amsmath}
\usepackage{amssymb}
\usepackage{amsmath}
\usepackage{amsfonts}
\usepackage{subfigure}
\usepackage{nonfloat}
\usepackage{url}
\usepackage[colorlinks=true, linkcolor=green, pagebackref]{hyperref}
\usepackage{textcomp} % for textonehalf
%\usepackage{subfig}
%\usepackage{subref}
%\usepackage{booktabs}

\usepackage{wrapfig}
\graphicspath{{imgs/}{data/renders_turn_table/}{imgs/rendered_results/}}

\usepackage{xspace}
\renewcommand*{\eg}{e.g.\@\xspace}
\renewcommand*{\ie}{i.e.\@\xspace}
\newcommand*{\ea}{et al.\@\xspace}
\renewcommand*{\vs}{vs.\@\xspace}
%\renewcommand{\arraystretch}{1.5}

%\cvprfinalcopy % *** Uncomment this line for the final submission

\def\iccvPaperID{####} % *** Enter the CVPR Paper ID here
\def\httilde{\mbox{\tt\raisebox{-.5ex}{\symbol{126}}}}

% General notation
\newcommand{\prob}{Pr}
\newcommand{\degree}{^{\circ}}
\newcommand{\feat}{\mathbf{x}}

% Image notation
\newcommand{\rgbdimage}{\mathcal{D}}
\newcommand{\intrinsics}{K}
\newcommand{\pixelidx}{\mathbf{s}}
\newcommand{\edgeimidx}{\mathbf{e}}

% Voxel notation
\newcommand{\voxelgrid}{\mathcal{V}}
\newcommand{\voxel}{v}
\newcommand{\voxidx}{i}
\newcommand{\voxelidxs}{m, n, l}

% Point cloud notation
\newcommand{\project}{\mathbf{p}}
\newcommand{\pcloud}{\mathcal{P}}
\newcommand{\point}{\mathbf{p}}
\newcommand{\normal}{\mathbf{n}}
\newcommand{\updir}{\mathbf{u}}

% Transformations
\newcommand{\trans}{T}
\newcommand{\extrinsics}{H}
\newcommand{\voxelgridtoworld}{\trans_{\voxelgrid \rightarrow w}}


\definecolor{red}{rgb}{0.95,0.4,0.4}
\definecolor{blue}{rgb}{0.4,0.4,0.95}
\definecolor{darkred}{rgb}{0.8,0,0}
\definecolor{darkgreen}{rgb}{0,0.5,0}
\definecolor{grey}{rgb}{0.6,0.6,0.6}

\newcommand{\todo}[1]{\textcolor{red}{TODO: #1}}
\newcommand{\note}[1]{\textcolor{blue}{NOTE: #1}}
\newcommand{\status}[1]{\textcolor{blue}{Status: #1}}
\newcommand{\add}[1]{\textcolor{darkgreen}{#1}}
\newcommand{\remove}[1]{\textcolor{grey}{#1}}

\renewcommand{\paragraph}{\vspace{2pt}\noindent\textbf}

%[citecounter=true, style=ieee]
%\usepackage{biblatex}
%\addbibresource{bibtex/strings.bib}
%\addbibresource{bibtex/main.bib}
%\addbibresource{bibtex/crossrefs.bib}
%addbibresource{\jobname.bib}


\title{Structured Prediction of Unobserved Voxels From a Single Depth Image}

%\author{Michael Firman, Gabriel Brostow, Simon Julier \ea}

\begin{document}


\maketitle

\begin{abstract}
	%Building a representation of the geometry of a scene is an essential task for many applications including robotic navigation, scene re-lighting and object manipulation.
	%Most existing works to recover the scene geometry rely on combining multiple views of the scene captured from many different directions or use of \emph{a priori} information about the expected semantic make-up of the scene.
  Building a complete 3D model of a scene is underconstrained given a single camera view.
  To gain a full volumetric model, one typically needs either multiple views, or a single view together with the strong assumption that one has a library of unambiguous training instances that fit the shape of each individual object in the scene.

  We hypothesize that objects of dissimilar semantic classes often share similar shape components, enabling a limited dataset to model the shape of a wide range of objects, and hence estimate the hidden geometry of various scenes.
  Exploring this hypothesis, we have implemented a system that can complete the unobserved geometry of a scene using a library of simple volumetric elements learned from training data.
  Using a novel feature representation, we train a model to map from observed depth values to an estimate of surface shape in the neighborhood of a 3D location.
  Through the prototype, we validate our approach qualitatively and quantitatively on a range of indoor scenes.

\end{abstract}

%%%%%%%%%%%%%%%%%%%%%%%%%%%%%%%%%%%%%%%%%%%%%%%%%%%%%%%%%%%%%%
\section{Introduction}
%\footnote{Potential to add to intro: Psychology: Are humans good at this? Not much evaluation in other work.}
%%%%%%%%%%%%%%%%%%%%%%%%%%%%%%%%%%%%%%%%%%%%%%%%%%%%%%%%%%%%%%


% What is the problem?
We broadly categorize space in our world as being `occupied' and opaque, or `vacant' and transparent.
Depth cameras such as the Microsoft Kinect are able to give an estimate of which regions of a scene are composed of free, vacant space.
However, each pixel in a depth image only makes an estimate of occupancy in front of the first solid surface encountered along that camera ray (Figure \ref{fig:intro}).
The `occlusion phenomenon' prevents any information from being measured about the occupancy of space beyond that first surface.

% Why is it interesting and important?
There are many applications, however, which critically require a complete representation of the world geometry.
When a robot sets out to grasp an unknown object in an unknown scene, a 3D model is required to get there and prevent collision with the object or nearby clutter.
Separately, in photo-editing, the full geometry would enable realistic shadows from a new light source to be automatically added to an image after it has been captured.

% Why is it hard? (E.g., why do naive approaches fail?)
Much computer vision research has been invested in reconstructing a full 3D world model from images of a scene captured from multiple viewpoints, thus coping with the effects of occlusion  (\eg \cite{izadi-uist-2011}).
Instead, we focus on the task of classifying each voxel in a 3D scene as being either `occupied' or `vacant' given just a single depth image from one viewpoint.
%% THERE WILL ALWAYS BE HOLES, EVEN WITH KINFU %%

\newcommand{\introsubwidth}{0.48\columnwidth}
\begin{figure}[!t]
    \centering
    \subfigure[Voxel world model]{%
        \includegraphics[width=\introsubwidth, clip=true, trim=110 105 205 30]{fig_1}}
        \hfill
    \subfigure[Depth rendering of world]{%
        \includegraphics[width=\introsubwidth, clip=true, trim=110 105 205 30]{fig_1b}}
    \caption{
    We model our world as a grid of voxels, each of which is either occupied or empty (vacant).
    An overhead view of a coarse 2D representation of this is shown in (a).
    When observed by a depth camera, only the first voxel along each ray is seen.
    This leaves a region of unknown occupancy extending beyond the depth surface (b).
    The aim of our algorithm is to predict the state of the voxels in this unknown region.}%
    \label{fig:intro}
\end{figure}

%%%%%%%%%%%%%%%%%%%%%%%%%%%%%%%%%%%%%%%%%%%%%%%%%%%%%%%%%%%%%%
\subsection{Our approach and contributions}
%%%%%%%%%%%%%%%%%%%%%%%%%%%%%%%%%%%%%%%%%%%%%%%%%%%%%%%%%%%%%%

Given a single depth image, our system predicts whether each voxel in the scene is occupied by a solid impermeable object, or free and empty.
In effect, we strive to predict the voxelized output of KinectFusion \cite{izadi-uist-2011}, but having test-time input of only a single view of the scene instead of multiple views.

We achieve this by learning a mapping from local and novel semi-regional features on a depth image to a structured prediction of geometry in the region of a query point, using a general-purpose collection of training objects and scenes.
We take inspiration from recent work which has segmented objects from images using silhouettes learned from different object classes \cite{kim-eccv-2012}.
This work showed that shape can transcend class categories, enabling shape predictions to be made regardless of the accuracy of semantic classifiers.
Because we care about shape, independently of semantic understanding, we are free to use training objects which differ from the objects being modeled in the scene.

The key contributions that underpin our novel depth image to voxel geometry framework are:
\begin{itemize}
\item \emph{Voxlets}, a representation of multi-voxel geometry in the region of a point in a scene.
We use a Random Forest to learn a mapping from a point in a depth image to a structured prediction of geometry in the region near the point.
\item We introduce a synthetic and a real dataset, setting a standard for evaluating scene completion algorithms.
The real dataset captures 50 desks with natural clutter.
\item We demonstrate the efficacy of a feature representation taken directly from a signed distance field.
This efficient representation effectively captures the geometry in a region of a depth image, and we believe it will have many further applications in segmentation and labelling.
\end{itemize}


\begin{figure*}
    \subfigure[Query point]{%
        \includegraphics[width=0.5\columnwidth, clip=true, trim=110 170 205 30, page=1]{overview_image}\label{subfig:voxregion}}
        \hfill
    \subfigure[Forest prediction]{%
        \includegraphics[width=0.45\columnwidth, clip=true, trim=80 170 265 30, page=2]{overview_image}\label{subfig:forest_overview}}
        \hfill
    \subfigure[Insertion into voxel grid]{%
        \includegraphics[width=0.5\columnwidth, clip=true, trim=110 170 205 30, page=3]{overview_image}\label{subfig:alignment}}
        \hfill
      \subfigure[Multiple predictions]{%
        \includegraphics[width=0.5\columnwidth, clip=true, trim=110 170 205 30, page=4]{overview_image}}
    \caption{\textbf{Overview of our algorithm, shown diagrammatically in a 2D representation.}
    (a) At test time, we define a cuboid region of voxels $\mathcal{R}$ around a query point $\pixelidx$.
    This cuboid is aligned with the normal at $\pixelidx$.
    (b) A feature representation $\mathbf{x}(\mathbf{s})$ is put through our pre-trained Random Forest.
    The forest makes a prediction for the values of each of the voxels in $\mathcal{R}$.
    (c) This prediction is transformed into the scene and used to update the values of the voxels.
    (d) The aggregation of multiple such predictions forms our final prediction of the TSDF, which can be converted to voxel occupancy or a surface representation.
    \label{fig:overview}
    }%
\end{figure*}
%%%%%%%%%%%%%%%%%%%%%%%%%%%%%%%%%%%%%%%%



%%%%%%%%%%%%%%%%%%%%%%%%%%%%%%%%%%%%%%%%%%%%%%%%%%%%%%%%%%%%%%
\section{Related work}
%%%%%%%%%%%%%%%%%%%%%%%%%%%%%%%%%%%%%%%%%%%%%%%%%%%%%%%%%%%%%%

%Most prior work in the area of completing missing data can be categorized according to its application domain (\eg meshes \cite{schnabel-eurographics-2009, ju-cst-2009}, 2D images \cite{gupta-cvpr-2011} or depth images \cite{shen-tog-2012}).


%\subsection{Taxonomy of related works}
Here we review related methods for completing unknown regions of visual data.
While similar, we do not cover the problem in 2D image completion.
Work in this area usually relies on the availability of extremely large amounts of similar images~\cite{hays-siggraph-2007} or on the assumption that the necessary structure for completion is present in the observed data~\cite{criminisi-cvpr-2003}.
Image completion typically aims for a visually plausible output, as opposed to accuracy compared to ground truth.
Additionally, our approach utilizes standard consumer hardware and as a result we do not review work that requires highly specialist equipment~\cite{velten-nature-2012}.


%%%%%%%%%%%%%%%%%%%%%%%%%%%%%%%%%%%%%%%%
\paragraph{3D primitives}\newline
`Geons' are proposed by \cite{bieberman-rbc-1987} as a set of 3D primitives such as cylinders and cuboids used by humans in their recognition of object shapes.
While in theory, geons could be used by computers  as features to describe natural objects, in practice, this was found to be challenging \cite{dickinson-iavc-1997} due to their ``idealized nature'', requirement for part segmentation, labeling errors, and the coarseness of features used to extract geons in the first place.
%Besl - hand-defined
However, fitting bounding boxes has recently become a popular method to explain the arrangement of objects in a scene.
Recent work has successfully incorporated high-level information such as gravity and stability
 \cite{shao-siggraphasia-2014, jia-cvpr-2013}, and made use of training data to accurately detect bounding box locations \cite{hedau-cvpr-2012}.
Gupta \ea \cite{gupta-cvpr-2011} estimate voxel occupancy from a 2D image, which is regularized using cuboid bounding box hypotheses.
The obvious problem with bounding box style methods is that they can only give coarse shape information, which is not suitable for many applications of geometry completion.

In our work, we make use of 3D primitives.
However, unlike geons which are fixed in shape, we learn a distribution of shapes from training data.
We are also able to make more fine-grained predictions than bounding boxes.


%%%%%%%%%%%%%%%%%%%%%%%%%%%%%%%%%%%%%%%%
\paragraph{Shape models}\newline
If prior knowledge is available about the objects present in the scene, in the form of 3D models, then an instance-level model can be fitted to a 3D scene.
This gives a good recovery of missing geometry of that object \cite{hinterstoisser-accv-2012, drost-3dimpvt-2012}.
Some works focus on the broader problem where an exact match is not present in the training set.
Shen \ea \cite{shen-tog-2012} complete the missing regions of a single object using an assembly of parts from several different models in a CAD database.
Cocias \ea \cite{cocias-cgvcv-2013} directly deform a mesh to fit to a point cloud, while Prisacariu and Reid \cite{prisacariu-iccv-2011} fit a class-level manifold model to the image data.
\note{Cite 3D ShapeNets CVPR 2015}

%Collecting training data for this is laborious and costly, and currently there do not exist enough 3D models suitable for accurately completing most real-world scenes.
All these methods, however, rely on the availability of some form of specific prior model, and on accurate detection to localize each object of interest in the scene.
In our approach, we set out to get as much shape information as possible \emph{without semantics}, thus remaining free of its associated machinery and limitations.


%%%%%%%%%%%%%%%%%%%%%%%%%%%%%%%%%%%%%%%%
\paragraph{Surface completion}\newline
Silberman \ea \cite{silberman-eccv-2014} tackle the completion of an incomplete multi-view reconstruction as a surface completion problem.
By detecting planes, they can complete their contours in a 2D projection using a novel CRF method.
This method, however, relies both on a piecewise-planar scene, and on beginning with an almost-complete scan as input.
Davis \ea \cite{davis-3dpvt-2002} complete surfaces by operating directly on the \emph{signed distance field}, the zero level-set of which defines the surface location. They diffuse the signed distance field across holes in the mesh to fill in the gaps.
\cite{harary-tog-2013} use a data-driven approach, finding matches in the mesh to the missing region.
Symmetry can be leveraged to complete some types of objects (\eg \cite{law-cviu-2010, thrun-iccv-2005, kroemer-humanoids-2012}).
However, this can be brittle, and if symmetry cannot be detected at all, then no prediction can be made.

All of these completion methods are only suitable where the set of missing data is small relative to the size of the observed data.
In our case, we are completing the geometry of an unknown surface which is typically at least as large as the observed surface.






%%%%%%%%%%%%%%%%%%%%%%%%%%%%%%%%%%%%%%%%
\paragraph{Voxel space reasoning}\newline
Finally, the two algorithms most similar to ours both make predictions of full scene geometry from a single depth image.
Kim \ea \cite{kim-iccv-2013} use a CRF model over a voxel representation of a scene to simultaneously predict occupancy, visibility and semantic labeling of voxels from an RGBD image.
For training, they use manually labeled top-down views of the scene.
They primarily model the probability of a voxel being occupied as a Gaussian centered on the first observed voxel along a camera ray.
However, high-order-terms in the CRF are used to enforce planar structures and for `objects' to remain contiguous.

Similarly,  Zheng \ea \cite{zheng-cvpr-2013} go from a single depth image to a voxel representation of a scene.
Their core algorithm consists of two parts.
Firstly they complete missing voxels by extruding visible points in the detected Manhattan World directions of the scene.
This is related to the completion method of \cite{kroemer-humanoids-2012}.
Secondly, they use physics-based reasoning to fill in missing data to ensure connectedness and stability.
The physics-based reasoning is clearly useful in complex scenes where, for example, supporting object parts may be occluded.
However the voxel completion by extrusion is limited by the Manhattan World assumption, and the extent of visible voxels in the scene.

In contrast to these methods, we make \emph{structured} predictions in 3D space, and reason about the space of shape variation using models learned from training data.
%we are able to complete objects where the points making up the objects are hard to see.}






% %%%%%%%%%%%%%%%%%%%%%%%%%%%%%%%%%%%%%%%%%%%%%%%%%%%%%%%%%%%%%%%%%%%%%%%%%%%%%%%%
\section{Approach formulation}
% %%%%%%%%%%%%%%%%%%%%%%%%%%%%%%%%%%%%%%%%%%%%%%%%%%%%%%%%%%%%%%%%%%%%%%%%%%%%%%%%


We model the geometry of a scene as a regular grid of voxels $\voxelgrid = \{\voxel_\voxidx\}$.
Following works such as \cite{izadi-uist-2011, prisacariu-iccv-2011}, each $\voxel_\voxidx \in [-d_{\max}, d_{\max}]$ represents the \emph{truncated signed distance function} (TSDF) of the surface of the scene.
Each $|\voxel_\voxidx|$ gives the distance from $\voxel_\voxidx$ to the nearest surface, truncated to a maximum value of $d_{\max}$, which is a parameter.
$\voxel_\voxidx$ is negative if voxel $\voxidx$ is inside solid opaque matter, and positive if it is in free space.
The zero level-set of $\voxelgrid$ therefore represents the surface.

\newcommand{\voxregion}{\mathcal{R}}

Our system maps a point $\pixelidx$ on an input depth image $\rgbdimage$ to a prediction of the TSDF in a set of voxels in the neighborhood of $\pixelidx_p$, which we use to denote the 3D reprojection of $\pixelidx$.
The aggregation of multiple such predictions gives our final TSDF prediction for the scene (Figure \ref{fig:overview}).

%$\project(\pixelidx)$, where $\project(\pixelidx)$ is the 3D location of $\pixelidx$ in world space.

%\paragraph{Voxel grid alignment}
%We make our predictions in a voxel grid aligned with world coordinates.
%In practice this means that the z direction of the voxel grid is aligned with the up direction of the world space.
%We do not make any Manhattan World assumptions so are free to position the voxel grid at any orientation about the z-axis.


\paragraph{Support regions}
The \emph{support region} $\voxregion \subset \voxelgrid$ is a set of voxels in the neighborhood of $\pixelidx_p$ for which our model can make a prediction of the TSDF.
Each $\voxregion$ is a fixed-size cuboid of voxels, whose x-axis is aligned with the normal direction at $\pixelidx$ (Figure \ref{subfig:voxregion}).

In a 2D world, the location of $\pixelidx$ and the direction of its normal can unambiguously define the location and orientation of $\voxregion$.
However, in 3D there is a degree of freedom unconstrained as the rotation of the cuboid about the axis of the normal is unspecified.
We resolve this by aligning the cuboid such that its z direction is coincident with the world z-axis, \ie the `up' direction of the scene.
The top and bottom face of each cuboid region $\voxregion$ is therefore parallel with the world's ground plane.

%The normal at $\pixelidx$ is used to constrain the degree of freedom around the z-axis.
%\todo{Figure for this!}
%The cuboid is then rotated about the z-axis so that its y-axis is perpendicular to the normal at $\project(\pixelidx)$.

\paragraph{Making a single prediction}
At test time, we extract a feature description of $\voxregion$, as described in Section \ref{sec:features}.
Using a structured Random Forest, we can make a prediction of the full geometry inside of $\voxregion$.
We call this prediction of geometry a \emph{voxlet}.
The voxlet, which comes out of the forest in canonical alignment, is then transformed from its local coordinate system into world space to fill the voxels in $\voxregion$.

The accumulation of multiple such predictions forms our final prediction of our TSDF.
This can then be converted to a prediction of surface geometry, described in Section \ref{sec:combining}.

%This transformation is the rotation $\Lambda$ followed by the translation $\mathbf{p}(\mathbf{\pixelidx}) - C(\mathcal{S})$, where $\mathbf{p}(\mathbf{\pixelidx})$ is the projection of $\pixelidx$ into world space, and $C(\mathcal{S})$ is the center of the voxlet.
%\note{This notation and explanation is a bit of a mess --- suggestions welcome! Also need to reference the figure in the text. Figure will ultimately be split into (a), (b) etc to make this easier.}


\paragraph{Training}
At training time, we similarly define a region $\voxregion$ around each point $\pixelidx$, again of a fixed size.
To help make our algorithm robust to different arrangements of objects we separate our ground truth grid into regions, and only extract ground truth voxels from the region associated with the point $\pixelidx$.
In this case, the ground truth values of $\voxelgrid$ and hence $\voxregion$ are known.
We use these ground truth TSDF values, together with the ground truth representations, to train a Random Forest, as explained in  Section \ref{sec:forest_train}.


\subsection{Pre-segmentation}

We note that while models such as PCA and Random Forests can model variation in shape on isolated objects using TSDFs well, modeling variations in arrangement of objects is much harder.
We quantified this using early experiments, which had good success in reconstructing objects in isolation, but struggled to complete objects in clutter.

We tackle this issue with objects in clutter both at training and and test time.
At training time, we assign a label to each voxel occupied voxel ($\voxel$ < 0) in $\voxelgrid$ such that  voxels with the same label are well connected to each other, and voxels with different labels are separated by some amount of free space.

When extracting a training voxlet from $\voxelgrid$, we note the label at $\pixelidx$.
We then only add to the training voxlet voxels which have the same label.
This ensures that each training voxlet typically models the shape of isolated objects.

Where objects are not well segmented at this stage, it is not a problem, as they will simply be used to make similar predictions at test time.

\todo{Segmentation method}

\note{Add images to make this more clear}

%%%%%%%%%%%%%%%%%%%%%%%%%%%%%%%%%%%%%%%%%%%%%%%%%%%%%%%%%%%%%%%%%%%%%%%%%%%%%%%%%
\section{Feature representation}
\label{sec:features}
%%%%%%%%%%%%%%%%%%%%%%%%%%%%%%%%%%%%%%%%%%%%%%%%%%%%%%%%%%%%%%%%%%%%%%%%%%%%%%%%%

\newcommand{\scalfeat}{x}

There have been recent successes in extracting simple features directly comparing depths at pixel locations in depth images, \eg \cite{shotton-cvpr-2011}.
While fast to compute, this feature is limited to making predictions for a single depth image.
In our case, we may have several images fused together into a common volume.\footnote{\note{We need to decide if we should tackle the multi-image case. This is more generalisable (sets us apart from some of the competition), but will need a few changes to the code base and paper text.}}
We therefore choose to extract features directly from the partially reconstructed signed distance volume $\voxelgrid$.

This has the benefit of being more invariant to camera viewing angle than features taken directly from the depth image, and given the voxel grid, it is computationally cheaper to extract features directly from the grid than to perform point cloud operations.

Our feature descriptor $\mathbf{x}$ consists simply of the raw TSDF values extracted at a set of pre-specified locations randomly sampled throughout $\voxregion$.
While values could be sampled from the full grid $\voxelgrid$ on-the-fly at the leaf nodes in the forest (similar to \eg \cite{shotton-cvpr-2011}).
However, we choose to extract $n$ such features from each $\voxregion$ in advance.

%The Random Forest--based model weights feature space to make structured predictions of the truncated signed distance function in the surrounding region.


% %%%%%%%%%%%%%%%%%%%%%%%%%%%%%%%%%%%%%%%%%%%%%%%%%%%%%%%%%%%%%%%%%%%%%%%%%%%%%%%%
\section{Learning a mapping from features to voxlets}
\label{sec:forest_train}
% %%%%%%%%%%%%%%%%%%%%%%%%%%%%%%%%%%%%%%%%%%%%%%%%%%%%%%%%%%%%%%%%%%%%%%%%%%%%%%%%
In this paper we pose unobserved geometry estimation given partial observed information as a supervised learning problem.
More specifically, our goal is to learn a function $f: \mathcal{X}\to \mathcal{Y}$, which maps each observed feature vector $\mathbf{x} \in \mathcal{X}$ to the label space $\mathbf{v} \in \mathcal{Y}$ representing the corresponding 3D geometry in the region $\mathcal{R}$ around $\mathbf{x}$.
Unlike standard classification where the goal is to predict a category label for each $\mathbf{x}$, our label space is a multi-dimensional structured vector $\mathbf{v} \in \mathbb{R}^{w\times{}d\times{}h}$ which encodes the TDSF values in a local region.
Inspired by the recent work of Doll{\'a}r and Zitnick~\cite{dollar-iccv-2013} we use a structured Random Forest to learn the function $f$.

\paragraph{Training}
To train the forest we pass a bagged subset of the training set $\left\{(\mathbf{x}_1, \mathbf{v}_1), ..., (\mathbf{x}_n, \mathbf{v}_n)\right\}$ to each node in the tree starting at the root.
Each node is then tasked with splitting the examples so that the ones sent to its children are as similar as possible in label space.
Instead of minimizing the structured loss directly, Doll{\'a}r and Zitnick~\cite{dollar-iccv-2013} approximate this loss at each node using a classification loss.
To convert the structured problem into a classification one we sample a different random subset of the dimensions of each $\mathbf{v}_i$ at the node, reduce their dimensionality to $N$ dimensions, and then cluster them into a discrete number of classes $K$ (here we set $N=20$ and $K=2$).
Then a standard classification loss can be used on this new discretization to evaluate the quality of different candidate splits for each $\mathbf{x}_i$.
In practice, this dimensionality reduction and clustering can be efficiently performed using randomized PCA~\cite{halko-siam-2011}.
To cluster, a training example is assigned to one of the two possible clusters $K$ based on the sign of the values in its first principal component.

This process is repeated until we cannot split the data any further.
Finally, each leaf node stores the medoid of all the examples that have arrived there, which we refer to as a voxlet, see Figure \ref{fig:voxlets}.

%$k$ = $min_k\sum_j(\mathbf{v}_{kj} - \bar{\mathbf{v}}_j)^2$

\paragraph{Testing}
Evaluating each tree at test time is extremely efficient.
For each location in the input depth image we simply traverse the tree until we reach a leaf node and return the voxlet stored at that location, see Figure \ref{subfig:forest_overview}.

\paragraph{Implementation}
Given the large dimensionality of $\mathcal{Y}$ we perform an initial dimensionality reduction to a 50 dimensional space using PCA.
Due to the large amount of redundancy in each $\mathbf{v}$ we found this to have little impact on the quality of our results, and yet it provides a large improvement to computational tractability.
(The effect of the PCA projection is quantified in Section \ref{sec:oracles}.)
We use an ensemble of 50 trees, which are grown to a maximum depth of 14 or until there is a minimum of 5 examples at a node.
We use simple axis aligned feature splits at each node.


\newcommand{\voxletsubwidth}{0.41\columnwidth}
\begin{figure}
     \includegraphics[width=0.32\columnwidth, clip=true, trim=110 30 110 30]{voxlets/pca_render_1.png}
     \includegraphics[width=0.32\columnwidth, clip=true, trim=110 30 110 30]{voxlets/pca_render_4.png}
     \includegraphics[width=0.32\columnwidth, clip=true, trim=110 30 110 30]{voxlets/pca_render_8.png} \\
     \includegraphics[width=0.32\columnwidth, clip=true, trim=110 30 110 30]{voxlets/pca_render_9.png}
     \includegraphics[width=0.32\columnwidth, clip=true, trim=110 30 110 30]{voxlets/pca_render_16.png}
     \includegraphics[width=0.32\columnwidth, clip=true, trim=110 30 110 30]{voxlets/pca_render_46.png}
     \caption{Some exemplar training voxlets extracted from the synthetic dataset (see Figure \ref{subfig:alignment}).
     Each voxlet can be seen to capture a section of the geometry of the scene.}
     \label{fig:voxlets}
\end{figure}



\pagedepth\maxdimen
\subsection{The voxlet dimensions}
\begin{wrapfigure}[10]{r}{0.35\columnwidth}
  \vspace{-30pt}
  \centering
    \includegraphics[width=0.40\columnwidth, clip=true, trim=160 180 440 120]{single_voxlet}
    \vspace{-15pt}
  \caption{}%The offset feature}
    \label{fig:voxlet_dims}
\end{wrapfigure}
The fixed-size regions $\mathcal{R}$ in the voxel grid are set in our experiments to have dimensions of $20 \times 40 \times 40$ voxels.
We make the voxlets longer in the y-direction as it is this direction that is approximately parallel to the normal at $\pixelidx_p$.
This allows the voxlet to make a larger prediction \emph{backwards} into the scene than \emph{sideways} into a region which typically already has observed data.
We set each voxlet to have real-world dimensions $0.1m \times 0.2m \times 0.2m$, meaning each voxel inside a voxlet has edge lengths of $(0.1 / 15)$m.
We define a local coordinate system for the voxlet, as shown in red in Figure \ref{fig:voxlet_dims}.
%The position in the voxlet which we align with the 3D point $\pixelidx_p$, as shown in Figure \ref{subfig:alignment}, is the point $x_v = y_v = z_v = 0.05$m.



%%%%%%%%%%%%%%%%%%%%%%%%%%%%%%%%%%%%%%%%%%%%%%%%%%%%%%%%%%%%%%%%%%%%%%%%%%%%%%%%
\subsection{Aggregating predictions}
\label{sec:combining}
%%%%%%%%%%%%%%%%%%%%%%%%%%%%%%%%%%%%%%%%%%%%%%%%%%%%%%%%%%%%%%%%%%%%%%%%%%%%%%%%

For each voxel in the output grid $\voxelgrid$, we accumulate the predictions of the multiple overlapping TSDF predictions at that location.
A number of different strategies could be employed to recover our final values, \eg filtering techniques, loopy belief propagation \etc.
In our work we simply take the mean of all the predictions at that voxel.
We note that the truncation of the signed distance function helps to make this style of accumulation robust.
A single incorrect estimation at a voxel can only be wrong by a maximum amount of $2d_{\max}$, where $d_{\max}$ is the level at which the distance function is truncated.

Some voxels in $\voxelgrid$ do not have any predictions made for them at all.
We assign to these voxels the value $d_{\max}$.

The final TSDF estimation can be converted to a mesh by finding the level-set of zero, for example using marching cubes \cite{lorensen-siggraph-2013}.


%%%%%%%%%%%%%%%%%%%%%%%%%%%%%%%%%%%%%%%%%%%%%%%%%%%%%%%%%%%%%%%%%%%%%%%%%%%%%%%%
\section{Experiments}
%%%%%%%%%%%%%%%%%%%%%%%%%%%%%%%%%%%%%%%%%%%%%%%%%%%%%%%%%%%%%%%%%%%%%%%%%%%%%%%%
We introduce two datasets to evaluate our algorithm, both of which will be placed in the public domain.

% \paragraph{Synthetic dataset D1}
Our \textbf{synthetic dataset (D1)} is a is a large dataset of 500 scenes, each featuring a random arrangement of CAD primitives.
These random but physically plausible arrangements were formed by dropping objects onto a plane using a physics simulator. Each scene is then rendered from 42 viewpoints in a hemisphere.
This large dataset allows us to easily demonstrate and quantify the effects of various parts of our algorithm.
As with our real data, the ground truth TSDF is computed by fusing together the depth images from all the cameras in the hemisphere.
We perform a 60/40 train/test split at a scene level.
At test time, one image is selected from the 42 input images for a test scene and reconstruction performed.

%\paragraph{Desks dataset D2}
Existing RGBD datasets of natural scenes typically focus on the per-frame labelling task, object detection or camera pose estimation.
There are no datasets to our knowledge which have the aim of capturing the full geometry of a scene.
We therefore introduce a \textbf{desktop dataset (D2)}, capturing the full geometry of **75** desktops in an office environment.
The desks were not manipulated before data capture, meaning they include the true intricacies of the real world.
This dataset has wide applications in geometry recovery, mesh segmentation and reconstruction.
We include the raw color and depth frames, together with the reconstructed mesh and voxelised volume for the data.


%%%%%%%%%%%%%%%%%%%%%%%%%%%%%%%%%%%%%%%%%%%%%%%%%%%%%%%%%%%%%%%%%%%%%%%%%%%%%%%%
\subsection{Experimental procedure}

In our experiments we first smooth the noisy depth data using bilateral filtering.
We then detect the largest approximately horizontal plane in reprojected 3D points, and use this to estimate the `up' direction of the scene.
We make an estimate of geometry using the 3D points which lie above this detected planar surface.

%A parameter of our  is how many pixels to evaluate.
We can choose at test time how many pixels to run through the forest to make a prediction of voxel occupancy.
In our experiments we use 200 pixels.
The forest predictions for these pixels are then aggregated as described in Section \ref{sec:combining}.



%%%%%%%%%%%%%%%%%%%%%%%%%%%%%%%%%%%%%%%%%%%%
\subsection{Quantitative evaluation}

\paragraph{Evaluation criteria}
The aim of our algorithm is to accurately classify free space around objects and in scenes.
Therefore, we report the per-voxel precision and recall over all the test data, using the sign of the accumulated TSDF to form our final binary prediction of occupancy.

We additionally report the area under the ROC curve, treating the predicted TSDF values as a soft prediction of occupancy.


%To find our test volume, we find the minimum volume axis-aligned bounding box for the ground truth voxel data before padding this voxel grid with 10cm of padding in each axis direction.


\paragraph{Baselines and algorithm variants}
\label{sec:algorithms}
We compare to two main baseline algorithms: a naive bounding box approach and the method of Zheng \ea \cite{zheng-cvpr-2013}.
The details of these baseline implementations, and of the details of variants of our method, are:

\noindent \textbf{(a) Bounding box} We fit a minimum-area bounding box to the 3D points belonging to the object, as defined by the ground truth object segmentation mask provided by \cite{singh-icra-2014}.
We are careful to remove `flying pixels', as they can have a large adverse effect on bounding box predictions.
Finally, the prediction of voxel occupancy is simply all voxels inside the bounding box are predicted to be occupied, while those outside are predicted to be empty.

\remove{The results from our quantitative analysis on the Bigbird turntable dataset are shown in Table \ref{tab:quant_results}.
We present some views of individual results in Figure \ref{fig:turntable_qual}.
We notice that the naive bounding box prediction suffers from a poor rate of recall.
This is expected, as typically the bounding box predictions under-predict the volume due to a lack of observed pixels to complete the shape of the object (\ref{fig:turntable_qual}).
Our results successfully capture the overall geometry of each of the objects.
The main, general point of failure is a smoothing effect which affects the edges of the objects.}

\remove{While our full forest outperforms the other methods, we notice only a small change in the variants of our approach using different features.
The lower precision and higher recall for the spider feature suggests that it is over-predicting the size of objects, although the performance is maximized when it is coupled with the offset feature.
The K-means Forest does almost as well as our full structured forest.
However, as it has discretized the label space before training we would not expect it to generalize well to larger and more diverse training data.}

\subsection{Introspection using oracles}
\label{sec:oracles}

A common question for algorithms like ours is: Where is the algorithm failing? We can gain some insight into the system by replacing various stages with an oracle that has access to the ground truth:

\textbf{OR1} is a modifed system which does not use any of the machine learning algorithm. Instead, at each $\mathcal{R}$ the ground truth voxels are extracted and then placed directly into the output grid. This system checks how good the coverage of voxlets is.

\textbf{OR2} is as OR1, except that the ground truth voxlets are compressed and then decompressed using the pre-learned PCA model. This system checks how well the PCA covers the range of voxlet shapes.

\textbf{OR3} is very similar to the full system, but in this system the machine learning algorithm which makes the voxlet predictions has access to the ground truth data. We use a nearest neighbours algorithm to find the closest training example to the ground truth voxlet at each location. The predictions are still restricted to be training examples.

\textbf{OR4} uses our Random Forest algorithm for prediction, but it uses the oracle at the aggregation step. Each voxlet is greedily added to the accumulator iff its inclusion would increase the AUC for this scene.


\begin{table}
  \centering
  \begin{tabular}{|p{2.2cm}|c|c|c|}
  \hline
  \textbf{Method}  &   \textbf{Precision} & \textbf{Recall} & \textbf{AUC} \\
  \hline
  OR1 & 0.882 & 0.958   & 0.933 \\
  OR2 & 0.879 & 0.959 & 0.945 \\
  OR3 & 0.878 & 0.951 & 0.942 \\
  OR4 & 0.819 & 0.968 & 0.928 \\
  \hline
  Full system & 0.798 & 0.939 & 0.898 \\
  \hline
  \end{tabular}
  \vspace{5pt}
  \caption{Qunatitiative results of our algorithm on the synthetic scenes. See section \ref{sec:oracles} for details.}
    \label{tab:oracle_results}
\end{table}





%%%%%%%%%%%%%%%%%%%%%%%%%%%%%%%%%%%%%%%%%%%%
\subsection{Qualitative scene results}
\label{sec:qualit}
\remove{In our qualitative analysis, we run the full structured prediction model from the previous section on views from the object segmentation dataset.
Qualitative results from this evaluation are shown in Figure \ref{fig:osd_qual}.
We note that the geometry of larger objects is recovered well, while the geometry of smaller regions can be missed.
Where objects are occluded and cluttered, our algorithm can fail to recover the most occluded sections.
See for example the result in row three, column three.}

%We believe that these results could be improved by
%This experiment demonstrates an ability for the algorithm to generalize from the turntable dataset.

\newcommand{\turnheight}{0.36\columnwidth}
\begin{figure*}
    \begin{tabular}{cccc}
\includegraphics[height=\turnheight]{umk4nke6pzebef2b_SEQ_input.png} &
\includegraphics[height=\turnheight]{umk4nke6pzebef2b_SEQ_visible.png} &
\includegraphics[height=\turnheight]{umk4nke6pzebef2b_SEQ_gt.png} &
\includegraphics[height=\turnheight]{umk4nke6pzebef2b_SEQ_pred_voxlets.png} \\
\includegraphics[height=\turnheight]{xf2hcoes8lp9fb1t_SEQ_input.png} &
\includegraphics[height=\turnheight]{xf2hcoes8lp9fb1t_SEQ_visible.png} &
\includegraphics[height=\turnheight]{xf2hcoes8lp9fb1t_SEQ_gt.png} &
\includegraphics[height=\turnheight]{xf2hcoes8lp9fb1t_SEQ_pred_voxlets.png} \\
\includegraphics[height=\turnheight]{kmrkmma8u2456lgk_SEQ_input.png} &
\includegraphics[height=\turnheight]{kmrkmma8u2456lgk_SEQ_visible.png} &
\includegraphics[height=\turnheight]{kmrkmma8u2456lgk_SEQ_gt.png} &
\includegraphics[height=\turnheight]{kmrkmma8u2456lgk_SEQ_pred_voxlets.png} \\
\footnotesize Input view of scene &
\footnotesize Input data in 3D space &
\footnotesize Ground truth occupancy &
\footnotesize Our reconstruction \
\end{tabular}
    \vspace{5pt}
     \caption{
     Qualitative results from the synthetic scene dataset. Each row shows the result of our algorithm on a different image from the database.
     We note that we can successfully recover the shape of many objects.
     It is in areas of occlusions, heavy clutter and specular objects that the algorithm misses areas (\eg the cylinder closest to the camera in row 3).
     \label{fig:osd_qual}
     }
\end{figure*}

\subsection{Limitations}
We can see that while we have success reconstructing well-observed objects, our algorithm can fail to recover geometry in occluded regions.
We propose that a physics-based-reasoning post-processing step, such as is proposed by \cite{zheng-cvpr-2013, shao-siggraphasia-2014}, would help the completion of ambiguous regions.

%Furthermore, by training on real-world scenes wo

%%%%%%%%%%%%%%%%%%%%%%%%%%%%%%%%%%%%%%%%%%%%
\section{Conclusions and future work}

In this paper we have presented an algorithm to successfully recover 3D geometry given just a single depth image.
Key to this approach is the voxlet, a set of voxels which can be learned from training data and then used at test time to recover shape.
%We valided this on

An interesting potential application of our method is to use the predicted completion as a prior for structure-from-motion.
Such a system would up its prediction of occupancy with observed data as it arrives from the capture device.
Additionally, using our prediction as a prior could form the basis of a \emph{next-best-view} algorithm \cite{Potthast2014148}.

%Currently we use a random sampling strategy to select the set of 3D points to use to form the reconstruction.
%We believe that a more intelligent sampling strategy, such as using the points which we believe to give accurate estimates of the depth \cite{reynolds-cvpr-2011}, may yield more accurate reconstructions.
%We would like to incorporate ideas from \cite{zheng-cvpr-2013, shao-siggraphasia-2014} by incorporating physics-based reasoning to help the completion of ambiguous regions.
%We also believe a coarse-to-fine version of our system would help to make the method more robust.



% \section{Acknowledgements}
% Peter Gehler
% Malcolm
% Neill
% Prism group
% Tom
% Peter
% Yotam
% Open source community - python scipy pcl etc

\pagebreak
{\small
\bibliographystyle{ieee}
\bibliography{bibtex/strings.bib,bibtex/main.bib,bibtex/crossrefs.bib}
}


\end{document}